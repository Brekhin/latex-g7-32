%% Преамбула TeX-файла

% 1. Стиль и язык
\documentclass[utf8x, 12pt]{G7-32} % Стиль (по умолчанию будет 14pt)

% Остальные стандартные настройки убраны в preamble.inc.tex.
\include{preamble.inc}

% Настройки листингов.
\ifPDFTeX
\include{listings.inc}
\else
\usepackage{local-minted}
\fi

% Полезные макросы листингов.
\include{macros.inc}

\begin{document}

\frontmatter % выключает нумерацию ВСЕГО; здесь начинаются ненумерованные главы: реферат, введение, глоссарий, сокращения и прочее.

% Команды \breakingbeforechapters и \nonbreakingbeforechapters
% управляют разрывом страницы перед главами.
% По-умолчанию страница разрывается.

% \nobreakingbeforechapters
% \breakingbeforechapters

%\include{00-abstract}

\tableofcontents

\include{10-defines}
\include{11-abbrev}

\Introduction

Видеоигры являются неотъемлемой частью индустрии развлечений. Изначально они были представлены в виде аркадных автоматов. С развитием технологий игры перекочевали на игровые консоли и персональные компьютеры. Так же с развитием сетевых технологий появились многопользовательские игры, ярким примером которых являются браузерные игры. 

Целью работы является создание браузерной игры "Понг".

\mainmatter % это включает нумерацию глав и секций в документе ниже

\chapter{Аналитический раздел}
\label{cha:analysis}
%
% % В начале раздела  можно напомнить его цель
%
В данном разделе анализируется и классифицируется существующая всячина и пути создания новой всячины. А вот отступ справа в 1 см.~--- это хоть и по ГОСТ, но ведь диагноз же...

\section{Анализ того и сего}

% Обратите внимание, что включается не ../dia/..., а inc/dia/...
% В Makefile есть соответствующее правило для inc/dia/*.pdf, которое
% берет исходные файлы из ../dia в этом случае.

\begin{figure}
  \centering
%  \includegraphics[width=\textwidth]{inc/dia/rpz-idef0}
  \caption{Рисунок}
  \label{fig:fig01}
\end{figure}

В \cite{Pup09} указано, что...

Кстати, про картинки. Во-первых, для фигур следует использовать \texttt{[ht]}. Если и после этого картинки вставляются <<не по ГОСТ>>, т.е. слишком далеко от места ссылки,~--- значит у вас в РПЗ \textbf{слишком мало текста}! Хотя и ужасный параметр \texttt{!ht} у окружения \texttt{figure} тоже никто не отменял, только при его использовании документ получается страшный, как в ворде, поэтому просьба так не делать по возможности.

\section{Существующие подходы к созданию всячины}

Известны следующие подходы...

\begin{enumerate}
\item Перечисление с номерами.
\item Номера первого уровня. Да, ГОСТ требует именно так~--- сначала буквы, на втором уровне~--- цифры.
Чуть ниже будет вариант <<нормальной>> нумерации и советы по её изменению.
Да, мне так нравится: на первом уровне выравнивание элементов как у обычных абзацев. Проверим теперь вложенные списки.
\begin{enumerate}
\item Номера второго уровня.
\item Номера второго уровня. Проверяем на длииииной-предлиииииииииинной строке, что получается.... Сойдёт.
\end{enumerate}
\item По мнению Лукьяненко, человеческий мозг старается подвести любую проблему к выбору
  из трех вариантов.
\item Четвёртый (и последний) элемент списка.
\end{enumerate}

Теперь мы покажем, как изменить нумерацию на «нормальную», если вам этого захочется. Пара команд в начале документа поможет нам.

\renewcommand{\labelenumi}{\arabic{enumi})}
\renewcommand{\labelenumii}{\asbuk{enumii})}

\begin{enumerate}
\item Изменим нумерацию на более привычную...
\item ... нарушим этим гост.
\begin{enumerate}
\item Но, пожалуй, так лучше.
\end{enumerate}
\end{enumerate}

В заключение покажем произвольные маркеры в списках. Для них нужен пакет \textbf{enumerate}.
\begin{enumerate}[1.]
\item Маркер с арабской цифрой и с точкой.
\item Маркер с арабской цифрой и с точкой.
\begin{enumerate}[I.]
\item Римская цифра с точкой.
\item Римская цифра с точкой.
\end{enumerate}
\end{enumerate}

В отчётах могут быть и таблицы~--- см. табл.~\ref{tab:tabular} и~\ref{tab:longtable}.
Небольшая таблица делается при помощи \Code{tabular} внутри \Code{table} (последний
полностью аналогичен \Code{figure}, но добавляет другую подпись).

\begin{table}[ht]
  \caption{Пример короткой таблицы с длинным названием на много длинных-длинных строк}
  \begin{tabular}{|r|c|c|c|l|}
  \hline
  Тело      & $F$ & $V$  & $E$ & $F+V-E-2$ \\
  \hline
  Тетраэдр  & 4   & 4    & 6   & 0         \\
  Куб       & 6   & 8    & 12  & 0         \\
  Октаэдр   & 8   & 6    & 12  & 0         \\
  Додекаэдр & 20  & 12   & 30  & 0         \\
  Икосаэдр  & 12  & 20   & 30  & 0         \\
  \hline
  Эйлер     & 666 & 9000 & 42  & $+\infty$ \\
  \hline
  \end{tabular}
  \label{tab:tabular}
\end{table}

Для больших таблиц следует использовать пакет \Code{longtable}, позволяющий создавать
таблицы на несколько страниц по ГОСТ.

Для того, чтобы длинный текст разбивался на много строк в пределах одной ячейки, надо в
качестве ее формата задавать \texttt{p} и указывать явно ширину: в мм/дюймах
(\texttt{110mm}), относительно ширины страницы (\texttt{0.22\textbackslash textwidth})
и~т.п.

Можно также использовать уменьшенный шрифт~--- но, пожалуйста, тогда уж во \textbf{всей}
таблице сразу.

\begin{center}
  \begin{longtable}{|p{0.40\textwidth}|c|p{0.30\textwidth}|}
    \caption{Пример длинной таблицы с длинным названием на много длинных-длинных строк}
    \label{tab:longtable}
    \\ \hline
    Вид шума & Громкость, дБ & Комментарий \\
    \hline \endfirsthead
    \subcaption{Продолжение таблицы~\ref{tab:longtable}}
    \\ \hline \endhead
    \hline \subcaption{Продолжение на след. стр.}
    \endfoot
    \hline \endlastfoot
    Порог слышимости             & 0     &                                                \\
    \hline
    Шепот в тихой библиотеке     & 30    &                                                \\
    Обычный разговор             & 60-70 &                                                \\
    Звонок телефона              & 80    & \small{Конечно, это было до эпохи мобильников} \\
    Уличный шум                  & 85    & \small{(внутри машины)}                        \\
    Гудок поезда                 & 90    &                                                \\
    Шум электрички               & 95    &                                                \\
    \hline
    Порог здоровой нормы         & 90-95 & \small{Длительное пребывание на более
    громком шуме может привести к ухудшению слуха}                                        \\
    \hline
    Мотоцикл                     & 100   &                                                \\
    Power Mower                  & 107   & \small{(модель бензокосилки)}                  \\
    Бензопила                    & 110   & \small{(Doom в целом вреден для здоровья)}     \\
    Рок-концерт                  & 115   &                                                \\
    \hline
    Порог боли                   & 125   & \small{feel the pain}                          \\
    \hline
    Клепальный молоток           & 125   & \small{(автор сам не знает, что это)}          \\
    \hline
    Порог опасности              & 140   & \small{Даже кратковременное пребывание на
    шуме большего уровня может привести к необратимым последствиям}                       \\
    \hline
    Реактивный двигатель         & 140   &                                                \\
                                 & 180   & \small{Необратимое полное повреждение
                                 слуховых органов}                                        \\
    Самый громкий возможный звук & 194   & \small{Интересно, почему?..}                   \\
  \end{longtable}
\end{center}

%%% Local Variables:
%%% mode: latex
%%% TeX-master: "rpz"
%%% End:

\chapter{Конструкторский раздел}
\label{cha:design}

В данном разделе проектируется новая всячина.

\section{Архитектура всячины}

\paragraph{Проверка} параграфа. Вроде работает.
\paragraph{Вторая проверка} параграфа. Опять работает.

Вот.

\begin{itemize}
\item Это список с <<палочками>>.
\item Хотя он и не по ГОСТ, кажется.
\end{itemize}

\begin{enumerate}
\item Поэтому для списка, начинающегося с заглавной буквы, лучше список с цифрами.
\end{enumerate}

Формула \ref{F:F1} совершено бессмысленна.

%Кстати, при каких-то условиях <<удавалось>> получить двойный скобки вокруг номеров формул. Вопрос исследуется.

\begin{equation}
a= cb
\label{F:F1}
\end{equation}


Окружение \texttt{cases} опять работает (см. \ref{F:F2}), спасибо И. Короткову за исправления..


\begin{equation}
a= \begin{cases}
 3x + 5y + z, \mbox{если хорошо} \\
 7x - 2y + 4z, \mbox{если плохо}\\
 -6x + 3y + 2z, \mbox{если совсем плохо}
\end{cases}
\label{F:F2}
\end{equation}

\section{Подсистема всякой ерунды}

Культурная вставка dot-файлов через утилиту dot2tex (рис.~\ref{fig:fig02}).

\begin{figure}
  \centering
% [width=0.5\textwidth] --- регулировка ширины картинки
%  \includegraphics{inc/dot/cow2}
  \caption{Рисунок}
  \label{fig:fig02}
\end{figure}


\subsection{Блок-схема всякой ерунды}

\subsubsection*{Кстати о заголовках}

У нас есть и \Code{subsubsection}. Только лучше её не нумеровать.

%%% Local Variables:
%%% mode: latex
%%% TeX-master: "rpz"
%%% End:

\chapter{Технологический раздел}
\label{cha:impl}

В данном разделе описано изготовление и требование всячины. Кстати,
в Latex нужно эскейпить подчёркивание (писать <<\verb|some\_function|>> для \Code{some\_function}).

\ifPDFTeX
Для вставки кода есть пакет \Code{listings}. К сожалению, пакет \Code{listings} всё ещё
работает криво при появлении в листинге русских букв и кодировке исходников utf-8.
В данном примере он (увы) на лету конвертируется в koi-8 в ходе сборки pdf.

Есть альтернатива \Code{listingsutf8}, однако она работает лишь с
\Code{\textbackslash{}lstinputlisting}, но не с окружением \Code{\textbackslash{}lstlisting}

Вот так можно вставлять псевдокод (питоноподобный язык определен в \Code{listings.inc.tex}):

\begin{lstlisting}[style=pseudocode,caption={Алгоритм оценки дипломных работ}]
def EvaluateDiplomas():
    for each student in Masters:
        student.Mark := 5
    for each student in Engineers:
        if Good(student):
            student.Mark := 5
        else:
            student.Mark := 4
\end{lstlisting}

Еще в шаблоне определен псевдоязык для BNF:

\begin{lstlisting}[style=grammar,basicstyle=\small,caption={Грамматика}]
  ifstmt -> "if" "(" expression ")" stmt |
            "if" "(" expression ")" stmt1 "else" stmt2
  number -> digit digit*
\end{lstlisting}

В листинге~\ref{lst:sample01} работают русские буквы. Сильная магия. Однако, работает
только во включаемых файлах, прямо в \TeX{} нельзя.

% Обратите внимание, что включается не ../src/..., а inc/src/...
% В Makefile есть соответствующее правило для inc/src/*,
% которое копирует исходные файлы из ../src и конвертирует из UTF-8 в KOI8-R.
% Кстати, поэтому использовать можно только русские буквы и ASCII,
% весь остальной UTF-8 вроде CJK и египетских иероглифов -- нельзя.

%\lstinputlisting[language=C,caption=Пример (\Code{test.c}),label=lst:sample01]{inc/src/test.c}

\else

Для вставки кода есть пакет \texttt{minted}. Он хорош всем кроме: необходимости Python (есть во всех нормальных (нет, Windows, я не про тебя) ОС) и Pygments и того, что нормально работает лишь в \XeLaTeX.

Можно пользоваться расширенным BFN:

\begin{listing}[H]
\begin{ebnfcode}
 letter = "A" | "B" | "C" | "D" | "E" | "F" | "G"
       | "H" | "I" | "J" | "K" | "L" | "M" | "N"
       | "O" | "P" | "Q" | "R" | "S" | "T" | "U"
       | "V" | "W" | "X" | "Y" | "Z" ;
digit = "0" | "1" | "2" | "3" | "4" | "5" | "6" | "7" | "8" | "9" ;
symbol = "[" | "]" | "{" | "}" | "(" | ")" | "<" | ">"
       | "'" | '"' | "=" | "|" | "." | "," | ";" ;
character = letter | digit | symbol | "_" ;
 
identifier = letter , { letter | digit | "_" } ;
terminal = "'" , character , { character } , "'" 
         | '"' , character , { character } , '"' ;
 
lhs = identifier ;
rhs = identifier
     | terminal
     | "[" , rhs , "]"
     | "{" , rhs , "}"
     | "(" , rhs , ")"
     | rhs , "|" , rhs
     | rhs , "," , rhs ;
 
rule = lhs , "=" , rhs , ";" ;
grammar = { rule } ;
\end{ebnfcode}
\caption{EBNF определённый через EBNF}
\label{lst:ebnf}
\end{listing}

А вот в листинге \ref{lst:c} на языке C работают русские комменты. Спасибо Pygments и Minted за это.

\begin{listing}[H]
\cfile{inc/src/test.c}
\caption{Пример — test.c} 
\end{listing}
\label{lst:c}

\fi

% Для вставки реального кода лучше использовать \texttt{\textbackslash lstinputlisting} (который понимает
% UTF8) и стили \Code{realcode} либо \Code{simplecode} (в зависимости от размера куска).




Можно также использовать окружение \Code{verbatim}, если \Code{listings} чем-то не
устраивает. Только следует помнить, что табы в нём <<съедаются>>. Существует так же команда \Code{\textbackslash{}verbatiminput} для вставки файла.

\begin{verbatim}
a_b = a + b; // русский комментарий
if (a_b > 0)
    a_b = 0;
\end{verbatim}

%%% Local Variables:
%%% mode: latex
%%% TeX-master: "rpz"
%%% End:

\chapter{Экспериментальный раздел}
\label{cha:research}

В данном разделе проводятся вычислительные эксперименты.
А на рис.~\ref{fig:spire01} показана схема мыслительного процесса автора...

\begin{figure}
  \centering
%  \includegraphics[width=\textwidth]{inc/svg/pic01}
  \caption{Как страшно жить}
  \label{fig:spire01}
\end{figure}


%%% Local Variables:
%%% mode: latex
%%% TeX-master: "rpz"
%%% End:


\backmatter %% Здесь заканчивается нумерованная часть документа и начинаются ссылки и
            %% заключение

\include{80-conclusion}

\include{81-biblio}

\appendix   % Тут идут приложения

\include{90-appendix1}
\include{91-appendix2}

\end{document}

%%% Local Variables:
%%% mode: latex
%%% TeX-master: t
%%% End:
