\chapter{Аналитический раздел}
\label{cha:analysis}

\section{Описание предметной области}

Понг является одной из самых ранних аркадных видеоигр, это теннисная спортивная игра с использованием простой двухмерной графики. Цель игры состоит в том, чтобы победить противника в настольный теннис, зарабатывая очки. Игра была создана фирмой Atari, которая выпустила её в 1972 году. Программист Аллан Алкорн создал Понг в качестве тренировки навыков, которую ему предложил Нолан Бушнелл. Бушнелл использовал идею пинг-понга, включенного в Magnavox Odyssey, что привело к иску против Atari\cite{WikiPong}.

Понг является простейшим симулятором настольного тенниса. Небольшой квадратик, заменяющий пинг-понговый мячик, двигается по экрану по линейной траектории. Если он ударяется о периметр игрового поля или об одну из нарисованных ракеток, то его траектория изменяется в соответствии с углом столкновения.

Геймплей состоит в том, что игроки передвигают свои ракетки вертикально, чтобы защищать свои ворота. Игрок получает одно очко, если ему удается отправить мячик за ракетку оппонента.

В Понг можно играть или одному человеку против компьютера, или вдвоем, когда каждый игрок управляет своей ракеткой.

\section{Анализ существующий решений}

На данный момент существует множество реализаций сетевого Понга.

\subsection{Atari Arcade PONG}

Браузерный Понг от Atari, создателя Понга.

\subsubsection{Преимущества}

\begin{itemize}
	\item Возможность однопользовательской игры.
	\item Возможность многопользовательской игры через приватные комнаты.
	\item Поддержка специальных контроллеров.
\end{itemize}

\subsubsection{Недостатки}

\begin{itemize}
	\item Реклама Internet Explorer.
	\item Приватные комнаты не работают.
\end{itemize}

\subsection{Flash Pong}

Существует множество реализаций многопользовательского понга на основе Flash технологии (Flash-игры). Описывать их по отдельности не имеет смысла, так как по своей сути они ничем не отличаются друг от друга. Примером таких игр является Multiplayer Pong от Kongregate\cite{KongreGatePong}.

\subsubsection{Преимущества}

\begin{itemize}
	\item Простое управление с помощью мыши.
	\item Пользовательский чат.
\end{itemize}

\subsubsection{Недостатки}

\begin{itemize}
	\item Устаревая flash-технология.
\end{itemize}

\section{Анализ предметной области}

Любая браузерная игра должна осуществлять поддержку всех актуальных на данный момент браузеров, таких как Google Chrome, Opera, Firefox и т.п.

\subsection{Определение объема и типов данных}

В многопользовательском понге должны быть реализованы следующие сущности:

\begin{itemize}
	\item Мяч
	\item Игрок
	\item Игровой движок
\end{itemize}

\subsection{Определение возможных действий пользователя}

Для пользователя должны быть доступны следующие действия (рисунок \ref{actions}):

\begin{itemize}
	\item Создание игровой сессии.
	\item Подключение к существующей игровой сессии.
	\item Управление ракеткой.
\end{itemize}

\begin{figure}
	\centering
	% [width=0.5\textwidth] --- регулировка ширины картинки
	\includegraphics[width=0.5\textwidth]{inc/pic/actions}
	\caption{Диаграмма прецедентов для пользователя}
	\label{actions}
\end{figure}


%%% Local Variables:
%%% mode: latex
%%% TeX-master: "rpz"
%%% End:
